% -*-LaTeX-*-
%
%    axis - A program for making two dimensional graphs
%

\documentstyle[11pt]{article}

% This is FULLPAGE.STY by H.Partl, Version 2 as of 15 Dec 1988.
% Document Style Option to fill the paper just like Plain TeX.
\topmargin 0pt
\advance \topmargin by -\headheight
\advance \topmargin by -\headsep

\textheight 8.9in

\oddsidemargin 0pt
\evensidemargin \oddsidemargin
\marginparwidth 0.5in

\textwidth 6.5in

\begin{document}

% For users of A4 paper: The above values are suited for american 8.5x11in
% paper. If your output driver performs a conversion for A4 paper, keep
% those values. If your output driver conforms to the TeX standard (1in/1in),
% then you should add the following commands to center the text on A4 paper:

% \advance\hoffset by -3mm  % A4 is narrower.
% \advance\voffset by  8mm  % A4 is taller.

%%%\endinput

% displayed text, indented, justification off
\def\displaybegin{\par\begingroup\medskip\narrower\narrower\noindent
		  \obeylines\obeyspaces}
\def\displayend{\endgroup\smallskip\noindent}

\newcommand{\bs}{$\backslash$}

%.nr PO 1.25i
%.nr HM 1.25i
%.nr FM 1.25i
%.nr LL 6.25i
%.po 1.25i
%.ll 6.25i

%.ND
\title{Axis - A program for making two dimensional graphs}
\author{}

\date{}

\maketitle

\section{General information}
Axis is a program for making two dimensional graphs.  It features
automatic scaling, logarithmic axes, error bars, and labels with Greek
letters, superscripts and subscripts, and special characters.
The output of axis is a file using a subset of the UNIX device independent
graphics commands, which can then be plotted on any device for which a 
plot filter is available.  Axis supports most of the options of the
UNIX graph program, but also contains many additional features.

Axis was intended to be easy to use for simple plots.  It should be
possible to produce a simple plot using no options or only a few
options.  For a finished plot, with nice labels etc., many options
may be necessary.

In the simplest case, the input to axis is a series of lines, each line
containing a pair of values. For example:
\displaybegin
1.0   1.0
2.0   2.1
3.0   2.9
\displayend

\noindent
With this input, axis will produce a graph with a line connecting all
the points.  The ranges of the x and y coordinates will be chosen
automatically to include all of the data.

There are a large number of options which modify the behavior of
axis.  These options may be specified either in the command line
of axis, or in lines in the input beginning with the '\#' character.
For example, to set the x range to be 0 to 1.5, you can use
\displaybegin
axis  -x  0  1.5  $<$  data  $>$  output
\displayend

\noindent
The minus sign on the x is optional, but it is a convention for UNIX arguments.
You can also include a line in the data file:
\displaybegin
\# x  0  1.5
0.0   1.0
1.0   2.0
2.0   3.0
\displayend

\noindent
(the points outside the range are ignored.)
There may be any number of option lines within the data file, and
they may appear at any point.  For most options, such as the x or y
ranges, it only makes sense to specify the option once.  If you
specify it more than once the last specification wins.  Specifications
in the command line win over specifications inside the file.
There are a number of options such as line style, plotting symbol,
or whether the data has error bars, which can be changed in mid plot.
Simply put an options line in the data file at the appropriate point.
For these options, the command line option gives the initial setting.

Axis can place a title on the graph, labels on the x and y axes, labels
at arbitrary points, or any character as a plotting symbol.  The character
set used in all of these labels is from the SLAC Unified Graphics
System.  It contains all the usual ASCII characters, Greek Letters,
Math symbols, and some special plotting characters.  The non ASCII
characters are specified by troff-like escape sequences. See below.

Axis is a descendant of the UNIX graph command.  Many of the options
are compatible with graph, although axis does require that
successive data points be on separate lines.  The output of axis
is in UNIX device independent graphics format (see plot(5)).
To display the results use the plot filters (plot(1)).

The remainder of this document contains information on the various
options to axis.  They are divided into options affecting the data
and how it is plotted and options affecting the labeling.  The distinction
is not clear.  Appendix A displays the UGS character set, appendix
B summarizes the escape sequences for special characters, and appendix
C summarizes the options.

\section{Options affecting the data plotting}
\subsection{x and y ranges, logarithmic scales}
An ``x'' or a ``y'' in the options string, followed by two numbers, forces the
x or y range to that given by the two numbers.  An ``x l'' or a ``y l''
makes the corresponding axis logarithmic, and if it is followed
by two numbers they are the range.
An optional third number following the x or y is the distance
between major tick marks along the axis.
For example
\displaybegin
x 0 2
x l
x l 1 1000
\displayend
are all legal options.

\subsection{Automatic abscissas}
If the ``-a'' option is specified the x coordinate is missing from
the data.  Axis will then plot points evenly spaced in x.  If a single
number is after the ``-a'', this number will be the spacing between
points, and if two numbers are present they will be the spacing
and the initial value of x.
The ``-a0'' option turns off the automatic abscissas.
These options may be turned on or off in midplot, or the abscissas
can be restarted at any desired value.  For example, to plot two traces
with spacing 2, starting at 0, use ``\#a 2 0'', then a list of ordinates
for the first trace, ``a 2 0'' to reset the abscissa, then the ordinates
of the second trace.

\subsection{Size and location of the graph.}
Axis by default makes a square graph almost filling the page.
To change the size of the box use ``-h number'', where the number is
the fraction of the default height, and ``-w number'' where the number
is the fraction of the default width.
To move the box, use ``-u number'' where the number is the fraction
of the box size to move up, or ``-r number'' to move right.
(It is usually necessary to move right a little if you want to
label the y axis. ``-r 0.1'' usually works.)

To rescale the entire graph, label sizes and all, use the ``-sc number''
option.  This rescaling is in addition to all rescalings done by
the ``-w'', ``-h'' or any of the label sizing options.  It also rescales
the length of the tick marks.

\subsection{Error bars}
A ``e'' (or a ``-e'') in an options string means that subsequent lines of
input should contain three numbers: an x value, a y value, and an error
on the y value.  Axis will then plot error bars.  The ``e'' option may
be changed in mid plot.  To turn off the
error bars and return to normal plotting, use an ``e0'' in an options
line.

\subsection{Linestyles}
A ``m number'' in the options changes the linestyle.  The recognized
values of the number are 0 for no lines (i.e. a point plot), 1 for
solid lines (this is the default), 2 for dotted lines, 3 for short
dashes, 4 for long dashes, and 5 for dot-dashed lines.  Not all output
devices will make all of these line styles.  This option may be 
changed in mid plot.

\subsection{Labeling points and breaking lines}
If the data in an input line is followed by a string in double quotes,
that string is used as a label for the point.  As noted before, the
character set contains a selection of plotting symbols which can be
used to plot the points.  To set a default label for each point
use the option 'c ``label''', and to turn a default label off
use ``c0''.  The label size can be adjusted by the option
``cs number''.  The default size is 1.5, in units of a terminal's
standard character size.
The label size option can be changed in midplot, so you can have
many different sizes of symbols in a graph.
See the section \ref{Labels} below on labels for
instructions on the use of the special characters.
This option is especially useful for plotting symbols, using the plot
symbols in the character set (cross, burst, etc.).

Any options line causes a break in the data.  That is, the point preceding
the options line will not be connected to the following point.
Thus an options line consisting only of a ``\#'' will cause a break.
The default action of axis is also to break the graph after each labeled
point or error bar, that is, not to connect it to the next point.  To force
connections, use the option ``-lb0''.
The option ``-lb1'' restores the default action of breaking after labels
or error bars.  (``-b'' is an archaic form of ``-lb0''.)
Another way to introduce a break is to use a label consisting of a blank
(still in double quotation marks!).

\subsection{Transposing x and y}
The ``t'' option transposes the x and y axes.  It doesn't work on
data with error bars.

\subsection{Superimposing graphs}
The ``-s'' option prevents axis from putting a screen clear at the
beginning.
This allows superimposing graphs, or when combined with the h,u,w, and
d options, allows more than one graph on the screen.

\subsection{Grid options}
Use the ``-g number'' option to set the grid style. 0 means no grid,
1 means a frame with tick marks, and 2 means a full grid.  The default
is 1.

\section{Labels} \label{Labels}
Axis will put titles on a graph, labels on the x and y axes, and
labels at arbitrary points.  There are a host of options to control
this.  All of the labels use the same character set.  Like all UNIX
strings, they must be enclosed in double quotes if they are more than
one word.  To get a real double quote into a label, use '\bs ``'.
Special characters are given by troff like escape sequences, which are
a backslash (``\bs '') followed by two characters.  For example, to
get a Greek letter use ``\bs gX'', which will produce the Greek equivalent
of the Roman letter X.  The case of the Greek letter will be the case
of X.  (The g may be either case, or you can use \bs *X for compatibility
with troff.)

Labels may have superscripts and subscripts.  To start a superscript
use ``\bs sp'', and to end a superscript use ``\bs ep''.  Use ``\bs sb'' and ``\bs eb''
to start and end a subscript.

To overstrike characters, underline characters, or fill a square root
sign, use ``\bs mk'' (mark) and ``\bs rt'' (return).  A ``mark'' remembers the
current point, and a ``return'' returns to it.  Thus to make a square
root sign with an ``a'' in it use the square root symbol, the overbar,
and the a: ``\bs sq\bs mk\bs rn\bs rta''

To get a real backslash into the label, use two backslashes. (``\bs \bs '')
The escape sequences are tabulated in appendix B.

\subsection{Specifying a title}
Use the 'lt ``title''' option to specify a title, where ``title'' is
a character string enclosed in double quotes.  If a title is given,
there are other optional options (sorry about that) to modify it.
``lts number'' modifies the title size, and the default size is 2.
``ltx number'' and ``lty number'' modify the placement of the title.
The default is top center of the graph, and the coordinate system used
is one in which the graph area runs from 0.0 to 1.0 in both the x and
y directions.

\section{Labeling axes}
The 'lx ``string''' and 'ly ``string''' options to put a label on the
x or y axis respectively.  The ``lxs number'' and ``lys number'' options
change the sizes of these labels.  The default axis label size is 1.5.
To change the size of the numerals on the scales for the x any y axes,
use the ``ls number'' option, again with a default size of 1.5.
``lxx number'',``lxy number'',``lyx number'', and ``lyy number'' change the
x and y coordinates of the x and y axis labels.

\section{Labels in the graph}
As noted above, a line of the form
\displaybegin
x\_value    y\_value    ``label''
\displayend
puts a label at the indicated coordinates (in the data's coordinate
system), and the ``-c label'' option may be used to set a default label,
or plotting symbol, for each point.  The ``cs number'' option changes
the label size.

\newpage
\appendix
\section{The UGS character set}

\newpage
\section{Escape sequences for special characters}
\par\vspace{1.0\baselineskip}
\par
%\begin{center}
\begin{tabular}{ll}
Escape sequence & Result (backslash omitted!) \\
 \\
\multicolumn{2}{l}{Plotting symbols} \\
pl & plus sign \\
cr & cross \\
di & diamond \\
sq & square \\
oc & octagon \\
fd & fancy diamond \\
fs & fancy square \\
fx & fancy cross \\
fp & fancy plus \\
bu & burst \\
 \\
\multicolumn{2}{l}{Greek letters (both cases)} \\
ga,gA & alpha \\
gb,gB & beta \\
gg,gG & gamma \\
gd,gD & delta \\
ge,gE & epsilon \\
gz,gZ & zeta \\
gy,gY & eta \\
gh,gT & theta \\
gi,gI & iota \\
gk,gK & kappa \\
gl,gL & lambda \\
gm,gM & mu \\
gn,gN & nu \\
gc,gC & xi \\
go,gO & omicron \\
gp,gP & pi \\
gr,gR & rho \\
gs,gS & sigma \\
gt,gT & tau \\
gu,gU & upsilon \\
gf,gF & phi \\
gx,gX & chi \\
gq,gQ & psi \\
gw,gW & omega
\end{tabular}
%\end{center}

%\begin{center}
\begin{tabular}{ll}
\multicolumn{2}{l}{Special symbols} \\
tm & times \\
dv & divide \\
+- & plus or minus \\
$<$= & less than or equal \\
$>$= & greater than or equal \\
\~{}= & approximately equal \\
n= & not equal \\
pt & proportional to \\
is & integral sign \\
li & line integral \\
pd & partial derivative \\
dl & del (gradient) \\
sr & square root \\
ul & underline \\
rn & overbar (``run'') \\
 \\
hb & h bar \\
lb & lambda bar \\
de & degree \\
in & infinity \\
dg & dagger \\
dd & double dagger \\
$<$- & left arrow \\
-$>$ & right arrow \\
da & down arrow \\
ua & up arrow \\
la & left angle \\
ra & right angle \\
ib & interabang \\
sc & section \\
 \\
ca & cap (intersection) \\
cu & cup (union) \\
mo & member of \\
nm & not a member of \\
ex & exists \\
al & for all \\
sb & subset \\
ds & direct sum (xor) \\
dp & direct product \\
sp & superset
\end{tabular}
%\end{center}

%\begin{center}
\begin{tabular}{ll}
\multicolumn{2}{l}{Control characters} \\
sp & start superscript \\
ep & end superscript \\
sb & start subscript \\
eb & end subscript \\
mk & mark location \\
rt & return to mark 
\end{tabular}
%\end{center}

\newpage
\section{Summary of options}
Square brackets indicate optional modifiers.  The minus sign is
unnecessary, just conventional.
\par\vspace{1.0\baselineskip}
\par
%\begin{center}
\begin{tabular}{ll}
-x [l] [xmin xmax] [xquant] & range of x, and logarithmic flag\\
-y [l] [xmin xmax] [yquant] & range of y, and logarithmic flag\\
-a [spacing] [initial] & automatic abscissas\\
-a0 & end automatic abscissas\\
-m style & linestyle\\
-g style & grid style\\
-s & don't erase\\
-e & data with errors\\
-e0 & end data with errors\\
-c ``string'' & default plot symbol\\
-c0 & turn off default symbol\\
-cs number & plot symbol size\\
-ls number & scale character size\\
-lb1 & break after labels\\
-lb0 & don't break at labels\\
-h number & fraction of height to use\\
-w number & fraction of width to use\\
-u number & move up\\
-r number & move right\\
-sc number & rescaling factor\\
-lt ``string'' & title\\
-lx ``string'' & x axis label\\
-ly ``string'' & y axis label\\
-lts number & title size\\
-lxs number & x label size\\
-lys number & y label size\\
-ltx number & title x coordinate\\
-lty number & title y coordinate\\
-lxx number & x label x coordinate\\
-lxy number & x label y coordinate\\
-lyx number & y label x coordinate\\
-lyy number & y label y coordinate
\end{tabular}
%\end{center}

\end{document}
